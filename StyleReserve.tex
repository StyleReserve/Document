\documentclass[conference]{IEEEtran}
\IEEEoverridecommandlockouts
% The preceding line is only needed to identify funding in the first footnote. If that is unneeded, please comment it out.
\usepackage{cite}
\usepackage{amsmath,amssymb,amsfonts}
\usepackage{algorithmic}
\usepackage{graphicx}
\usepackage{textcomp}
\usepackage{multicol}
\usepackage{multirow}
\usepackage{xcolor}
\def\BibTeX{{\rm B\kern-.05em{\sc i\kern-.025em b}\kern-.08em
    T\kern-.1667em\lower.7ex\hbox{E}\kern-.125emX}}
\begin{document}

\title{StyleReserve\\
{\footnotesize \textsuperscript{}- LG Styler Reservation Service that protects your Styler usage, clothes for special occasions, and your own style -}
}

\author{\IEEEauthorblockN{1\textsuperscript{st} Kim TaeHyeon}
\IEEEauthorblockA{\textit{Dept. of Information Systems} \\
\textit{Hanyang University}\\
Seoul, Republic of Korea \\
kimtahen@hanyang.ac.kr}
\and
\IEEEauthorblockN{2\textsuperscript{nd} Park SeEun}
\IEEEauthorblockA{\textit{Dept. of Information Systems} \\
\textit{Hanyang University}\\
Seoul, Republic of Korea \\
seeunplayer@gmail.com}
\and
\IEEEauthorblockN{3\textsuperscript{rd} Lee SeungHo}
\IEEEauthorblockA{\textit{Dept. of Information Systems} \\
\textit{Hanyang University}\\
Seoul, Republic of Korea \\
all100v@hanyang.ac.kr}
\and
\IEEEauthorblockN{4\textsuperscript{th} Heo JeongYoon}
\IEEEauthorblockA{\textit{Dept. of Information Systems} \\
\textit{Hanyang University}\\
Seoul, Republic of Korea \\
jyheo8221@gmail.com}
}

\maketitle

\begin{abstract}
Our team - Dressing Sauce - is focusing on the Styler-using environment inside the family and workspaces on the expansion, and plan to develop a Reservation System that helps to reduce conflict in the process of using the Styler. On top of that, we also planned a Calendar-based function that helps to reserve what I want to wear in advance, which helps to prepare clothes the day before appointment.

The "StyleReserve - LG Styler Reservation Service" came to our mind as we thought about ways to solve problems we can face in our daily lives, such as differences in time of returning home between family members, or when we have to use Stylers in an emergency. Based on the type of clothes the user want to wear, the functionality of the Styler they use, and the reservation time, family members can save time and effort by using the Styler together and can provide more people with the opportunity to wear neat clothes. 

In addition, ‘StyleReserve’ can provide a seamless experience of using an LG Styler in an office, such as workplace attendance or after dinner during overtime, in order to promote people in the company to use LG Styler fairly and efficiently.

We also wanted to create a service that provides a "notification" so that users can "book their clothes in advance" and notify them to use Styler before the appointment. For example, in the case of sisters living together, we found out that they often share their clothes together. In this situation, if there is a system in which one can reserve clothes for his or herself in advance on the calendar, then it will prevent small conflicts because of the cases when two people have to wear the same clothes on the same day.

Furthermore, if the system can send a notification the day before the user has a special appointment in order to encourage him or her to use the Styler, we expected this can help users to go out nicely.

In conclusion, our team hopes to provide a calendar-based reservation system to help families and office workers experience using Stylers, and to further enhance the functionality of existing Stylers by helping users to wear neat clothes on important days.

\end{abstract}

\begin{IEEEkeywords}
Styler, Reservation System, Calendar, Mobile Application, Clothes
\end{IEEEkeywords}

\section {Role Assignment}
Below is the table of role descriptions. Due to the small members in the team, some roles are distributed to the same person in order to harmoniously progress the project.

\begin{table}[htbp]
\begin{tabular}{| p{1cm}|p{1.5cm}|p{5.2cm} |}

\hline
Name & Roles & Role Description\\

\hline

\end{tabular}
\end{table}

\begin{table}[htbp]
\begin{tabular}{| p{1cm}|p{1.5cm}|p{5.2cm} |}

\hline
Kim Tae Hyeon & Development Manager & A Development Manager is responsible for managing the overall project. He ensures that the project is progressing well on schedule, that the functions are being implemented well, and that the overall software engineering process of development, UI/UX design, and prototype testings and supplementation are going smoothly. In addition, he continuously monitors whether team members are communicating smoothly and completing the project together.\\

\hline
Park Se Eun & Customer, Product Designer & A Customer thinks about what features they want to add when using the product. For example, in the process of using an LG Styler, after experiencing the moment when it becomes difficult to use because others are already using it, the customer soon reach to a need for the function in these kinds of situations, and intends to use this idea in the process of developing the LG  Styler using environment later.

A Product Designer is responsible for designing applications to provide the most efficient interface from the user's point of view. She works with developers to think about how to reflect users and deliver functions most efficiently to them.\\

\hline
Lee Seung Ho & Software Developer & A Software Developer is responsible for developing software. Based on the needs of Users and Customers, he plan the details of the services that will be effective for them.

In addition, these plans are developed later so that they can be realized through code implementation. Moreover, he keeps track on the function by continuously checking whether these codes are operated effectively based on feedback from users and consumers.\\

\hline
Heo Jeong Yoon & User & A User considers the need for new services based on various phenomena that can be experienced in everyday life. For example, in the process of living with her sister and sharing her closet together, she often encountered situations where she had to wear the clothes her sister wanted to wear on the same day. She soon had a need for the existence of a service that could solve this problem. Her needs were positively received by the team, who were considering LG Styler's innovative functions.

Since then, she has consistently contributed to the completeness of the service by continuously testing the application, delivering and reviewing feedback on what is lacking and to be supplemented. \\
\hline

\end{tabular}
\end{table}

\section{Introduction}
\subsection{Motivation}
1) Consumer Trend Analysis
\begin{itemize}
    \item Increased external activities due to the easing of COVID-19: Due to the lifting of distancing from the easing of COVID-19, many people are enjoying outside life such as festivals and dinners again, and also interest in clothes, which is the basis for enjoying outside and out-of-home life, is growing. According to the research in 2020, the consumption of bags arose from -32.1 percent to 38.1 percent, and clothes from -15 percent to 17.4 percent respectively, showing that people are prepared to go out.\\
    \item Spreading demand for convenience and hygiene: Based on the trend called ‘Pyunrimium’ which is the collaboration of ‘Pyunri’ meaning convenience and ‘Premium’, clothing managers that can easily manage clothes at home without going to a laundry are also attracting consumers' attention. In addition, as consumers' sentiment to keep their clothes clean spreads due to the influence of COVID-19 and fine dust in Korea, the clothing manager market is rapidly growing, and the clothing manager itself is becoming a home trend.\\
    \item Predicting a Continuous Increase in Demand for Clothing Managers: According to the "Home Appliance Trend Report 2022" released by market research firm Open Survey, Clothing Managers are listed as the No. 1 home appliances that people hope to purchase within the next year. In addition, according to a survey by ‘Daehak Nail’, a famous Korean research institute especially focusing on young trends, both Generation Z and Millennials selected "clothing managers" as household appliances that wants to purchase and replace in the future, and they are especially highly willing to buy Stylers among various clothing managers. This was judged to be a long-term positive demand for clothing management, not a short-term.\\
    \item In the process, we decided that Stylers, who were in high demand for suits and other clothing, will be loved by Generation Z, and as a team with Generation Z's eyes, we thought about how the product can provide attractive functions to all members of society.\\
\end{itemize}

2) Company analysis
\begin{itemize}
    \item Expanding Artificial Intelligence: LG Electronics is adding a large number of new AI services to LG ThinQ applications this year and is trying to provide innovation in customer experience by systematically designing AI development strategies.\\
    \item UP-home appliances: LG Electronics is developing a variety of software with the aim of providing a more advanced smart home experience and an upgrade experience that allows users to receive good services without buying home appliances.\\
\end{itemize}

3) Product analysis
\begin{itemize}
    \item LG Styler History: LG Styler, the flagship of the nation's clothing appliances, is the first clothing management device launched in 2011, and is loved by many people for providing a system that can manage clothes at home without much time and labor. In particular, as the concentration of fine dust in Korea and the COVID-19, people started to realized the need for cleanliness and now interest in Stylers is growing more explosively.\\
    \item Consumers' Advantages of Styler: Based on various key patents, it provides effective clothing management that differentiates itself from other brands.\\
    \item Consumers' Hope for Styler: As Samsung, a latecomer brand in this market, tries to gain the upper hand by focusing on software features since they failed to have efficient patent, there is a small regret in terms of providing additional software features for LG Styler.
    \begin{figure}[htbp]
        \centerline{\includegraphics[width=8cm]{삼성엘지기능비교.png}}
        \caption{Comparison of Clothing Management Functions of LG and Samsung}
        \label{fig}
    \end{figure}
    
\end{itemize}
As a result, the team wants to propose a new software from a consumer perspective that reflects the trend of LG Electronics' "UP-home appliances" which LG Electronics is currently pushing for, and solves even minor inconveniences for consumers using Stylers.\\

\subsection{Problem Statements}
First of all, we were able to hear the story from an acquaintance of Generation Z and an Internet post that she was angry because her sister wore her clothes on the day when she planned to wear for the special day in advance. This case was frequently confirmed in a house with sisters or in a family where sisters live together.

  \begin{figure}[htbp]
        \centerline{\includegraphics[width=7cm]{언니동생싸움.jpg}}
        \caption{Example of the problem found on the SNS}
        \label{fig}
    \end{figure}

\newpage
In response, we thought of a system that uses a pre-built closet database within a styling calendar to restrict others from wearing clothes on the same date if someone booked clothes on a specific date already. On top of that, we also thought of a function to induce users to wear neat clothes the next day by sending a notification of the use of an LG Styler the day before the scheduled date.

A total of five tops and one bottom can be placed in the Styler, and the function of the Styler, which runs depending on the type of clothing, takes as little as 30 minutes to as much as 2 hours. Considering the limited number of clothes that can be put in the Styler and the commuting time, our team has come to think that we won't have much time to use the Styler at home on weekdays.

We thought of a situation that we had to use the LG Styler in a hurry within the family unit, and that other people might already be using it. If users are going to use the same course, they can use the Styler together, but since communication was not smooth, they have to activate the Styler twice, wasting time and power. As a result, our team came up with a reservation system that allows other families to use the Styler together by taking clothes from a pre-built closet database and help them to book the use of Styler if there are any seats left. The application also displays a timeline for using the Styler, allowing users to conveniently check and take the available time zone of the Styler.

From this point of view, the problem that occurs in offices that share Stylers is more clear. More people use it together than using at home, which sometimes causes multiple people to flock at the same time. In addition, the Styler is very likely to operate inefficiently when used by multiple individuals, as the capacity of the Styler is limited while the course execution time of the Styler is fixed. Our Styler reservation system can solve these problems intuitively and clearly. If such a function is provided in the future, the use environment of the Styler can be expanded not only at home but also to public places such as offices.

In addition to the main functions mentioned above, we would like to create an LG Styler service as a true stylist that helps styling A to Z, considering the ultra-personalized customization service that recommends clothes for users.\\

\subsection{Research on Any Relative Software}
\begin{itemize}
    \item Space cloud: The Space Cloud application provides the ability to freely reserve, share, and use the space you need on an hourly basis. Users can select the desired space among various categories (meeting/practice/shooting/shared office, etc.) in real time, enter the scheduled time, number of people or purposes to use together, and reserve a place rental. On the reservation history page, you can see the status of your reservation at a glance. However, the reservation details are not visible to the people who are planning to use the space together, so there is a regret that the person who reserved has to inform the reservation details in another way.\\
    
    \item Naver Calendar: The service created by Naver provides a function to set my schedule and display it on the calendar. In addition, various additional functions such as attaching stickers, decorating colors, and setting goals are provided, so the demand is high for Generation Z who prefer to decorate their own diary.
    Above all, the web and mobile environment are highly interconnected, so the schedule created on the mobile can be synchronized immediately and checked on the PC. You can also mark the weather forecast for a week on your calendar for reference.\\
    \item To-Do Mate: To-Do Mate is an application that combines scheduler and SNS(Social Network Service) functions and provides the ability to share and communicate one's schedule with others. By organizing the checklist and completing the daily schedule or checklist, users can express empathy and praise with emojis such as "Like", and can see the progress of their schedules and routines at a glance, providing motivation effects. In particular, it also provides an alarm function to remind you if you have an important or forgotten schedule.\\
\end{itemize}

\section{Requirement Analysis}

\subsection{The ability to log in or create an account}
 If the application is not logged in, family members who use the Styler together can create their own account and log in. 
\begin{itemize}
    \item Sign up: You can log in by setting your email and password.
    \item Sign in: Move to the main page.\\
\end{itemize}

\subsection{Linking with Styler}
A logged in account can add an LG Styler to its own account using the Styler's QR code and password. Several people can register as users in the same Styler, so the service is available for families or small members. When you click on the Styler icon, the styling calendar is displayed on the screen.\\

\subsection{Closet database construction function}
You can configure the closet database by LG Styler. Based on Musinsa, the most well-known fashion platform centered on the MZ generation, information on clothes is retrieved and stored in a database. Because one Styler provides the same database, users who use the same Styler use the same database in the same closet. Once registered, the application can continuously check the clothes, so there is no need to register again.
\begin{itemize}
    \item Enter the product number in the input window of the screen to get information about clothes from Musinsa, the most well-known fashion platform centered on the MZ generation.
    \item If the item number is not known, the user may directly enter information on the name and item, and store it in the closet.\\
\end{itemize}

\subsection{Styler Reservation Calendar}
Styler Reservation Calendar: The Styler Reservation Calendar allows users to view the status of their Styler Reservation by event for the selected date, allowing them to create a new reservation or add clothing to the remaining space in the existing reservation.\\
\begin{itemize}
    \item The Styler assumes that you can reserve five sets by default. Therefore, regardless of the number of reservations, you can proceed with the reservation by adding clothes until 5 clothes are registered.\\
    \item If there are other appointment times or Styler courses that you would like, you can reserve the Styler through a new appointment.\\
    \item If you have all the clothes you want for other reservations at the time you want, you cannot make a reservation at that time.\\
\end{itemize}

\subsection{Styling Reservation Calendar}
You can import and reserve the clothes you want to wear on a specific day in the closet database. Since the reservation of clothes is made by date, no one else can reserve the same clothes on the same day. When you schedule clothes on a specific date, the previous day's notification prompts users to use the Styler.\\
\begin{itemize}
\item
When you select the clothes you want to wear in your closet, the dates you can and cannot make a reservation appear on your calendar.
\item
If you have clothes to wear on December 16th and you don't have any other reservations on that date, you can confirm your reservation by entering the reason why you want to wear them on that date.
\item
On December 16, the day before December 15, users will be notified to put their pre-booked clothes into the Styler.\\
\end{itemize}

\subsection{Check Styler Usage}
Users can see at a glance what course they have booked based on today's date and what is the most recently scheduled styling schedule on the main screen.\\

\section{Development Environment}
\subsection{Choices of Software Development Platform}
Our main development environment is Windows. StyleReserve provides a reservation system that adjusts the schedule so that you can use it when you want in an environment where you jointly use LG Stylers such as home and company. As this service is designed to increase convenience in the process of using existing products as part of the function provided by LG ThinQ, we intend to produce it in the form of a mobile application. 

In this project, we intend to develop the back-end of applications using Node.js runtime and express.js web framework, and create applications using the React Native library and check progress through Expo. And the UI/UX design of the application is to be carried out using Figma. The overall establishment, execution, and process management of the project schedule utilize Notion, and related documents are shared through Google Drive and the project is carried out.

\begin{table}[htbp]
\caption{Development Language and Environment}
\begin{tabular}{| p{1.7cm}|p{6.4cm} |}
    \hline
    Tools and Language & Reason \\
    \hline
    Javascript(JS) & Javascript was originally used mostly in the front-end field for the purpose of implementing dynamic web pages, but since Google's V8 javascript engine-based node.js runtime was developed to enable javascript to run in a non-browser environment, it has become very scalable. Expanding from the existing front-end field, many fields including back-end and applications can be developed through this language.
    In this project, we want to build a backend and build an application using node.js runtime. Since javascript has a low entry barrier, it is possible to implement the service without difficulty. Therefore, we chose this language to focus on the design, design, and implementation of the service, regardless of the difficulty of the programming language in this project.\\
    \hline
\end{tabular}
\end{table}

 \begin{table}[htdp]
 \caption{Team's Work Environment}
 \begin{tabular}{| p{3cm}|p{5.2cm} |}
 \hline
 Name & Environment \\ 
 \hline
 Heo JeongYoon & Windows 11 Home with 64-bit Operating System, and x64-based Processor \\
 \hline
 Kim TaeHyeon & Arch Linux 64-bit \\ 
 \hline
 Lee SeungHo & Windows 7 Professional K \\ 
 \hline
 Park SeEun & Window 10 Pro with 64-bit Operating System, and x64-based Processor \\ 
 \hline
 \end{tabular}
 \end{table}

\begin{itemize}
    \item Cost Estimation: To implement StyleReserve, it is necessary to obtain information from the server while obtaining data from the database or communicating with the server in real time. Therefore, servers that are hosted in real time or multiple APIs are required. However, we tried to secure open APIs and free servers as much as possible in the current project of developing prototypes as students, and in this process, we were able to actively utilize various tools that were allowed freely for students for a specific period of time.\\
    Currently, free servers are used for prototype development purposes, but if the service has a plan to be officially implemented, problems are likely to arise as server traffic increases and copyright problems arise. Therefore, in the process of official service launch, it is estimated that the need for additional payment will emerge as the payment of paid services is required.\\
    
    \item Using Any Commercial Cloud Platform: This project uses Amazon Lightsail as a backend server and database. AWS Lightsail is a virtual private server service launched by Amazon that provides low prices and vast amounts of traffic. It has the advantage of simple and clear setup screens such as databases and networking, and can control charging time. Currently, students with student certification are given free access to the service for three months, so it has been decided to use it in this project to develop a prototype model.\\
    \end{itemize}

\subsection{Software in use}
\begin{figure}[htbp]
\centerline{\includegraphics[width=2.5cm]{ReactNative.png}}
\label{fig}
\caption{Logo of React Native}
\end{figure}
1) React Native: React Native is a JavaScript framework that allows users to build applications running on iOS and Android and is based on React, a library of JavaScript from Meta used to develop web front-end. Most of the written code can be used equally between platforms, so it has a great advantage in that it can develop iOS and Android apps simultaneously using React native. In addition, given that strong developer tools and meaningful error messages are basically included in the framework, we decided to use them because we believed that the app could be effectively implemented in a short time.\\

\begin{figure}[htbp]
\centerline{\includegraphics[width=3cm]{Expo.png}}
\label{fig}
\caption{Logo of Expo}
\end{figure}
\newpage
2) Expo: Expo is a build tool for developing React Native as a cross-platform, making modules easier to use and test on real-world devices quickly. In order to check whether the prepared React Native code works smoothly, we installed Expo on the mobile phone, and then ran expo on the VSCode to check and proceed.\\

\begin{figure}[htbp]
\centerline{\includegraphics[width=4cm]{postgresql.png}}
\label{fig}
\caption{Logo of PostgreSQL}
\end{figure}
3) PostgreSQL: In this project, the data must have a complex relationship with each other and continue to be modified. Although the NoSQL database shows very good performance compared to RDBMS, we decided to use RDBMS because we need to ensure the integrity of the data relationship rather than performance in this project. We chose PostgreSQL among RDBMS.
PostgreSQL is an open source object-relational database system (ORDBMS). It was developed based on BSD licenses, so unlike commercial DBMS, there is no license fee. As an open source dbms, stability is guaranteed due to its long development period and continuous versioning, and it supports numerous extensions and data types. In particular, it has a great advantage in that it continues to develop because of its steady updates. Currently, the popularity and demand of PostgreSQL continue to increase, and in this regard, we have decided to use PostgreSQL in our project.\\

\begin{figure}[htbp]
\centerline{\includegraphics[width=4cm]{Nodejs.png}}
\label{fig}
\caption{Logo of Node.js}
\end{figure}
\newpage
4) Node.js / Express.js: Node.js is a javascript runtime built with Google's v8 javascript engine that provides an environment for javascript code execution in a desktop environment away from the browser. There are several other web frameworks based on this express.js as well as express.js in the web framework running on Node.js runtime. Using the express.js framework, we can build a compact and lightweight RESTful api. Therefore, in this project, we decided to use the express.js web framework, which operates at node.js runtime, for back-end construction.\\

\begin{figure}[htbp]
\centerline{\includegraphics[width=4cm]{lightsail.jpeg}}
\label{fig}
\caption{Logo of AWS Lightsail}
\end{figure}
5) AWS Lightsail: Lightsail is a virtual server provided by AWS, which has the advantage of being able to use a single physical server independently of a large number of users at a low cost.\\

\begin{figure}[htbp]
\centerline{\includegraphics[width=4cm]{Figma.png}}
\label{fig}
\caption{Logo of Figma}
\end{figure}
6) Figma: Figma is a web-based tool that allows anyone to access the same file and exchange feedback with each other with a shared link, regardless of the installation file or OS. Therefore, it was judged as the most effective tool to be used in the project, where everyone had to constantly check the results and share feedback. In addition, as Figma provides convenience in design and color change or shadow effect application, it was judged that it will help a lot in effective UI/UX configuration.\\

\begin{figure}[htbp]
\centerline{\includegraphics[width=5cm]{Github.png}}
\label{fig}
\caption{Logo of Github}
\end{figure}
7) Github: Github is a storage role of distributed version management system. It is a public code web host management system to effectively manage source code web host management system.
As possible to track changes in computer files, it is effective to prevent confusion between multiple users, it serves to prevent confusion in the process of collaboration.The process such as simple file upload and download and download and downloading the task can be easily through version management tools.\\

\begin{figure}[htbp]
\centerline{\includegraphics[width=5cm]{Notion.png}}
\label{fig}
\caption{Logo of Notion}
\end{figure}
8) Notion: Notion is software that enables users to manage project progress as a whole as well as personal records. The overall concept planning, function planning, and development of this StyleReserve project were managed through Notion by actively utilizing the advantage that all team members can access and modify the contents of the same page using Team Space.\\

\begin{figure}[htbp]
\centerline{\includegraphics[width=5cm]{overleaf.png}}
\label{fig}
\caption{Logo of Overleaf}
\end{figure}
9) Overleaf: Overleaf is an online LaTeX editor that provides a variety of templates that fit the concept, making documentation easy and available on the web without the need to install them separately. Overleaf is fast, easy to find errors or typos, and easy to share, making it helpful to use especially during team projects. In addition, you can immediately check the results that reflect what you wrote through the code, making it easier to work.\\

\subsection{Task Distribution}
\begin{table}[htbp]
\caption{Task Distributions for Each Member}
\begin{tabular}{| p{1cm}|p{1.5cm}|p{5.2cm} |}

\hline
Tasks & Name & Description\\

\hline
\end{tabular}
\end{table}

\begin{table}[htbp]
\begin{tabular}{| p{1cm}|p{1.5cm}|p{5.2cm} |}

\hline
Front-end Developer & Heo JeongYoon, Park SeEun, Kim Taehyeon & The front-end developer is simply the job of developing everything the user sees. Since t it is the first part of the customer's search and use of service information, front-end can also be defined as the first impression of the application. We check the responses of customers and users, determine how to find solutions quickly.
Front-end developers work on business logic configuration and user interfaces through output and input of data imported from the back-end application programming interface (API). In this respect, the front-end can be seen as serving as a bridge between the back-end and the planning.
In general, people who have a lot of experience using services from the user's point of view, such as the web and apps, and who can read and reflect trends are likely to do well.\\

\hline
Back-end Developer & Kim TaeHyeon, Lee SeungHo & The back-end developer is responsible for implementing the overall business logic of all services, analyzing and improving existing systems, and continuing research and learning about new systems. Back-end developers can develop a service's business logic and gain an overall experience with web services through database integration and architecture. With a lot of interest in databases and DB management and system architectures, you can even gain experience to expand your knowledge through various platforms, infrastructure, and network experiences. Although risks to Incidents may be born in the process of development, it is judged that risks can be minimized through sufficient development verification, vulnerability check, code review, and CI. In addition, these plans are developed later so that they can be realized through code implementation. Moreover, he keeps track on the function by continuously checking whether these codes are operated effectively based on feedback from users and consumers.\\

\hline
UI/UX Designer & Park SeEun, Heo JeongYoon & The UI/UX designer is responsible for service planning and design using methodology, which is the first step in app front development. This job performs an in-depth analysis of the service from the planning stage and repeat steps such as user research, hypothesis verification, and wireframe design to make the service more complete. Before launching the web and app services, flow verification and usability testing using wireframes are conducted, and afterwards, prototypes are produced based on improved flows, and design is carried out in earnest.
Distinguish UI and UX design tasks in detail, UX designers are interested in how they feel about the product and try many different approaches to solve specific user problems. The most important responsibility for UX designers is to ensure that the product moves from one stage to the next. Unlike UX designers, UI designers are more interested in how to place products. UI designers are responsible for designing each screen and page so that users can interact properly based on UX, as well as creating a style guide and applying consistent design representations according to the product. \\
\hline

\end{tabular}
\end{table}

\section{Specifications}
StyleReserve provides users a variety of reservation features that they need while using the LG Styler. When a user who is stored in the database accesses the application, three functions are given as follows.

\begin{figure}[htbp]
\centerline{\includegraphics[width=9cm]{서비스흐름도.png}}
\caption{Overall Process}
\label{fig}
\end{figure}

Even in an environment with multiple people using Styler, the reservation system help people to reserve the Styler at any time they want by receiving data by users.
In addition, when the user enters the cloth's information, the information is stored in the closet database, and informs the user to use the Styler before the scheduled date if the user set some specific date of wearing specific cloth, inducing the user to wear clean and neatly managed clothes on important days.\\

\begin{figure}[htbp]
\centerline{\includegraphics[width=9cm]{MenuStructure.png}}
\caption{Screen Structure}
\label{fig}
\end{figure}
This is a flowchart of the overall screen that constitutes the application. After logging in, the user can meet the screen classified according to each function on the checkable Main Page, and use the desired functions on each screen.\\

\subsection{Logo}
\begin{figure}[htbp]
\centerline{\includegraphics[width=2cm]{applogo.png}}
\caption{StyleReserve Logo}
\label{fig}
\end{figure}
\newpage
0) Logo: StyleReserve's logo is expressed as a model of a hanger, which implies a field related to fashion. The hanging part is consist with the alphabet ‘S’ in order to contain the first letter of the service name ‘Style’, and the bottom part is expressed by putting the letter of Reserve as it is, so that the name of the service can be expressed as much as possible just by looking at the logo.\\

\subsection{Entry}
\begin{figure}[htbp]
\centerline{\includegraphics[width=3cm]{FlashScreen.png}}
\caption{Flash Screen}
\label{fig}
\end{figure}
1) Flash Screen: This screen is the page that you see when you first turn on the app. By inserting a lettering hanger design that shows the identity of the app, it allows users to feel more familiar with the app. The overall background is filled with the color "#a50034", which is the signature color of LG. Through the 'Flash Screen', the users can imagine LG and feel the brand more closer.\\

\begin{figure}[htbp]
\centerline{
\includegraphics[width=3cm]{login-1.png}
\includegraphics[width=3cm]{login-2.png}
}
\caption{Login Screen}
\label{fig}
\end{figure}
\newpage
2) Login Screen: This login screen has a 'Sign-in' button, 'Sign-up' button and a 'Find Password' button. When the login button is pressed, the information in the ID and PASSWORD box is transmitted to the back-end, and the back-end checks the user's information in the database to check if it matches and sends a response according to the success of the login. If the login is successful, switch the screen to the main page. If the login fails, the message 'ID or PASSWORD is invalid' is displayed.\\

\begin{figure}[htbp]
\centerline{
\includegraphics[width=3cm]{signup.png}
\includegraphics[width=3cm]{signup-success.jpg}
}
\caption{Sign-Up Screen}
\label{fig}
\end{figure}
2-1) Sign-Up Screen: The person who uses the Styler together can create an account by pressing the 'Sign-Up' button.
\begin{itemize}
    \item Sign-Up: During the signing-up process, the user needs the username, email for id that the user is going to use, password, and password again to confirm.
    \item Sign-Up Success: If new information that does not overlap with the existing Username and Email is received as an input value, a notification window appears with the message "You successfully signed up" referring the user's Sign-Up was successful. If the user succeed in signing up for a member, then the user can return to the login screen through the button and sign in through the information entered when signing up.\\
\end{itemize}

\begin{figure}[htbp]
\centerline{
\includegraphics[width=3cm]{findpassword.png}
\includegraphics[width=3cm]{passwordresult.jpg}
}
\caption{Find Password Screen}
\label{fig}
\end{figure}
2-2) Find Password Screen: If the user forgets password, then user can go to the 'Find Password' page and can check what the password was. 
\begin{itemize}
    \item Insert Information: When the user enters the username and email and click the 'Find password' button, the password is send to the email the user wrote.
    \item Go back to Login: A button appears with a message that the password the user entered at the time of signing up has been sent to the e-mail address. The button allows the user to return to the login screen. Users can check the password from the email address they entered with Username, and can log in.\\
\end{itemize}

\subsection{Main Page}
\begin{figure}[htbp]
\centerline{\includegraphics[width=3cm]{mainpage.png}}
\caption{MainPage Screen}
\label{fig}
\end{figure}
1) Main page: Successful login will leads the user to go to the main screen. When the main screen turns on, the two biggest features - 'Schedule Styler' and 'Schedule Styling' - can be seen, and if the user presses that button, then the user will be taken to the page where each function is available.
In addition, on the main page, the page shows the most recently planned Styler and Styling schedule. Users can check the information at a glance without having to go to another page.
From the main screen, the navigation bar is fixed within the application, so users can efficiently navigate to the page they want.
\begin{itemize}
    \item Recent Reservation: It displays the schedule closest to the current time. The Styler reservation section contains the courses user wish to use with a big letter, and the date and time of the reservation and the estimated course time are written at the bottom. The styling reservation shows the reservation for the nearest date of the upcoming reservation, stating the date on which the clothes were booked, the purpose for which the clothes were booked, and whether you would receive the previous day's alarm.
    \item Navigation Bar: This is designed to go directly to Main Page, Styler Reservation, Styling Reservation, My Page. Users can click the icon for each page to go directly to the page they want.\\
\end{itemize}

\subsection{Styler Schedule Reservation}
The Styler Reservation is a function that allow users to reserve LG Styler in the situation where multiple people share a single Styler together. The database stores information about courses, reservations, people booked together, and information about the number of clothes that are stored at each date and time. On the screen, when a user selects a desired reservation date from the calendar, all reservation information scheduled for that day is retrieved and displayed, and a new reservation, or additional reservation, is encouraged.\\

\begin{figure}[htbp]
\centerline{\includegraphics[width=3cm]{Today's reserve.png}}
\label{fig}
\caption{Today's Reservation Screen}
\end{figure}
1) Today's Reservation: When the user enters Styler Reservation page, the user will see the reservation for today's date by default. At the top, the calendar shows a week, and if the user drags down the calendar, the user can scroll through the monthly calendar and schedule a reservation.
If the user select a specific date, then the user can check the scheduled time of the course and the designated time for each course on the left. On the right, there are information about which course people booked, who firstly booked special reservation, how many clothes are currently reserved in a five-piece Styler, and who is going to use the LG Styler at that time. All users of the same LG Styler can check that information.\\

\begin{figure}[htbp]
\centerline{\includegraphics[width=3cm]{addreserve.png}}
\label{fig}
\caption{Add to Existing Reservation Screen}
\end{figure}
2) Styler Reservation in case of hangers left: If there is ① an existing reservation on the date and time that the user wants to use, ② the reserved course is that the user wanted, and ③ there are seats left inside the styler, then the reservation can be made by adding the number of clothes to the existing Styler reservation. It increases efficiency by preventing overlapping with other users when users try to make a reservation at a specific time or allowing them to manage clothes together if the types of clothes are the same.
Since the user adds a reservation with all the situations set, the reservation can be added by simply entering the number of clothes. By pressing the "Add Booking" button, the user can confirm that his or her reservation has proceeded correctly as the number of clothes reserved for the existing reservation and the list of reservations are modified.\\

\newpage
\begin{figure}[htbp]
\centerline{\includegraphics[width=3cm]{alreadydonereserve.jpg}}
\label{fig}
\caption{Reservation with No Vacancy Screen}
\end{figure}
3) Styler Reservation in case of reservation full: If all five clothes are gathered and filled in a reservation at a specific time, the reservation details will have a gray background and the button to add the reservation disappears, making it impossible to modify. In this case, no matter how much the user wished to make a reservation for that time, the user has no choice but to select a different time to proceed with the reservation.\\

\begin{figure}[htbp]
\centerline{
\includegraphics[width=2.5cm]{newreserve.png}
\includegraphics[width=2.5cm]{settingtime.png}
\includegraphics[width=2.5cm]{stylerreserve.png}
}
\caption{Styling Calendar Reservation}
\label{fig}
\end{figure}
4) Styler Reservation in case of making new reservation: After selecting a specific date, when users press the "Make a new reservation" button at the bottom of the calendar, a small pop-up window appears. The pop-up window is designed to set the time to reserve, the course to be reserved, and the number of clothes to be reserved. If the user adjusts the contents well and confirm the reservation, the reservation appears as a new box, and if the reservation is continuously added on each day, users can check the reservations on the day at once through scrolling.\\

\subsection{Styling Schedule Reservation}
In addition to booking a Styler, users can also use a system to reserve clothes to use for styling. Styling reservation provides a key function to book clothes that must be worn on important dates such as meetings or weddings on the application, and to receive notifications the day before that date to induce Stylers to be used. Through the notification, the user may act to wear organized clothes on an important day.\\

\begin{figure}[htbp]
\centerline{\includegraphics[width=3cm]{overallstylingreservation.jpg}}
\label{fig}
\caption{Overall Styling Reservation}
\end{figure}
1) Overall Styling Schedule: When the user moves to the Styling Reservation Section, the first calendar that deals with Overall Styling Reservation appears. The calendar displays all reservations for clothes registered in the closet. The mark can be checked whether the dot below the number of the calendar is marked.

\begin{figure}[htbp]
\centerline{\includegraphics[width=3cm]{mycloset.jpg}}
\label{fig}
\caption{My Closet Screen}
\end{figure}
2) My Closet Screen: Users can store their clothes in a database built within the Styler and then select them from their closet, and schedule the use of the Styler whenever there is a special schedule.\\

\begin{itemize}
    \item Save New Clothes: Users can store clothes in their closets by selecting a product name, brand name, classification of clothes, etc. When the user clicks ‘Save’ button, the contents are delivered to the closet database and saved. Users can check the stored clothes on the ‘My Closet Screen’ and then proceed with the process when they need to make a reservation for the use of a Styler.
    
    \begin{figure}[htbp]
    \centerline{
    \includegraphics[width=3cm]{clothreserve.jpg}
    \includegraphics[width=3cm]{clothreserve_reason.jpg}
    \includegraphics[width=3cm]{clothreserve_success.jpg}
    }
    \label{fig}
    \caption{Cloth Reservation}
    \end{figure}
    \item Book Clothes schedule: To book a cloth schedule, users must have the dress registered in his or her closet by default. Under the condition that clothes are registered in the closet, users can press the clothes they want to make a reservation to go to the styling reservation screen.\\
\end{itemize}

3) Styling Reservation Information Screen: As seen on the Styler Reservation Screen, if the user selects a specific date from the calendar, the clothing information reserved on that date appears. Cloth reservations are designated by date without a concept of time, so multiple people cannot book clothes on a single date. Users can check who made a reservation for the use of the clothes and what is the purpose of the reservation, and if a reservation is not made on the date selected by the other user, the user can proceed with the reservation for the use of the clothes on that date. All reservations are shared among users because people who use the same Styler have the same database, thus preventing mishaps of wearing the same clothes on the same day in advance.\\

\newpage
\begin{figure}[htbp]
\centerline{\includegraphics[width=3cm]{Cloth reservation.png}}
\label{fig}
\caption{Appointment Information Screen}
\end{figure}
\begin{itemize}
    \item Clothes Appointment: The appointment date is fixed for the date of the user’s choice. The user enters the reason why the user is trying reserve the clothes.
\end{itemize}
    
\begin{figure}[htbp]
\centerline{\includegraphics[width=3cm]{Notification Center.png}}
\label{fig}
\caption{Notification}
\end{figure}
\begin{itemize}
    \item Styler Usage Notification: Notifications are delivered to users that induce them to use the Styler the day before they are scheduled to wear clothes. The notification can be found on the Notification Page located on the top bar. The notification states that the user has reserved clothes for some purpose and therefore needs to use a Styler. The notification allows users to use a Styler and go to an important schedule the next day in neat clothes.\\
\end{itemize}

\newpage
\subsection{My Page}
\begin{figure}[htbp]
\centerline{\includegraphics[width=3cm]{My Page.png}}
\label{fig}
\caption{My Page Screen}
\end{figure}
The user can check the username that the user set and the email entered in the My Page Screen. In addition, the system logout button is also on the page, so if the user want to log out of the application, the user can proceed through 'My Page'.\\


\section{Architecture Design & Implementation}
\subsection{Overall Architecture}
Our service consists largely of three modules: application, database, and server.
\begin{figure}[htbp]
\centerline{\includegraphics[width=8cm]{Application Framework.png}}
\label{fig}
\caption{Application Framework}
\end{figure}

The first module is the front-end. To make the service more accessible to users, we designed and implemented React Native and JavaScript-based applications. The calendar within the application allows users to see the status of their Styler reservations, especially when available and when not available through color separation. Users can also check the current status at a glance with Styling reservation that runs from the closet to the reservation page. In addition, users can check it out through push notifications when a schedule approaches.\\
The second module is the back-end that interacts with the database. We used node.js, JavaScript, and express.js to build a back-end that delivers information to the front-end. In our database, we have a table that manages the information of users using the application, a table that manages the user's Styler usage reservation status, a table that builds a single closet system based on clothing data, and a table that manages the reservation status of clothing for styling reservations. Our database was created by postgreSQL, and the back-end receives the user account, stores it in the database, and verifies that the user is still using the application.\\

\begin{figure}[htbp]
\centerline{\includegraphics[width=8cm]{erd.png}}
\label{fig}
\caption{Database Entity Relationship Diagram}
\end{figure}
The third module is the server. We've built a server with AWS Lightsail to allow the overall process to proceed. Node.js contains code that allows users to respond to requests generated while using the application, and allows the server to generate query statements based on the parameters of the request to read or write data.\\

\subsection{Directory Organization}
StyleReserve is composed with three Github repositories, which are StyleReserve front-end, StyleReserve back-end, and Document. In StyleReserve front-end repository, files related to the overall design and fuctions to interact with users of the application is in.
In StyleReserve back-end repository, it has files that work with the repository and database.
The Document repository, it includes Latex code and PDF file of the document of this project.
\begin{figure}[htbp]
\centerline{\includegraphics[width=7cm]{repository.png}}
\label{fig}
\caption{Repository}
\end{figure}

\subsection{Module 1: Front-End}
1) Purpose: We wanted to create a React Native-based application to help users experience their Styler. React Native can develop a cross-platform and has many libraries and functions to utilize, making it easy to develop. It can also be developed by checking the actual execution screen through Expo. Through the application, we connect the user to the server and store the value entered by the user to the back-end. It also receives data from databases provided by the back-end and allows users to view it.\\

2) Functionality: React Native allows the application to communicate with each object between the user and the back-end server through the UI/UX and to perform the main functions of the application. It created a screen to show users through front-end work with React Native, and implemented major background service functions such as signing up, logging in, registering Styler usage reservations (date, time, course, number of clothes), registering clothes information, and registering clothes wearing reservations (date, purpose). Meanwhile, the server and the database can be connected to the application to obtain user account data and related information.\\

3) Location of Source Code:StyleReserve/StyleReserve-Frontend \\

\begin{table} [h]
    \begin{tabular}{p{2cm}|p{3cm}|p{2.5cm}}
    \hline
    \textit{\textbf{Directory}} & \textit{\textbf{File Name}} & \textit{\textbf{Modules Used}}\\ \hline
    \begin{tabular}[c]{@{}l@{}}StyleReserve/\\StyleReserve\\-Frontend\end{tabular} & \begin{tabular}[c]{@{}l@{}}App.js\\app.json\\babel.config.js\\babel.config.js\\babel.config.js\\package.json\end{tabular} 
    & \begin{tabular}[c]{@{}l@{}}Frontend (React Native)\end{tabular} \\\hline
    
    \begin{tabular}[c]{@{}l@{}}StyleReserve\\-Frontend\\/assets/\end{tabular} & \begin{tabular}[c]{@{}l@{}}All_icon.png\\Styler_img.png\\adaptive-icon.png\\add-red.png\\add.png\\addpic.png\\back-arrow.png\\big-black-leather.png\\black-leather.png\\blue-dress.png\\bottoms_icon.png\\brown-mustang.png\\check.png\\clock.png\\close.png\\closet.png\\closet_fill.png\\clothes-hanger-white.png\\clothes-hanger.png\\color-notification.png\\down-arrow.png\\dress.png\\dresses_icon.png\\favicon.png\\flitz.png\\group-red.png\\group-white.png\\group.png\\hanger2.png\\hawai-shirts.png\\home.png\\home_fill.png\\icon-button2.png\\icon-button3.png\end{tabular} 
    & \begin{tabular}[c]{@{}l@{}}-\end{tabular} \\\hline
    \end{tabular}
\end{table}

\begin{table}[h]
    \begin{tabular}{p{2cm}|p{3cm}|p{2.5cm}}
    \hline
    \begin{tabular}[c]{@{}l@{}}StyleReserve\\-Frontend\\/assets/\end{tabular} & \begin{tabular}[c]{@{}l@{}}icon.png\\icon5.png\\image-2.png\\left-arrow.png\\logo-grad.png\\logo-white.png\\mainpage-table.png\\mainpage-tshirt.png\\multiple.png\\navy-slacks.png\\notification-black.png\\notification.png\\outline.png\\pants.png\\pink-dress.png\\pink-hoodie.png\\researchicon.png\\reserve.png\\shirt.png\\small-black-leather.png\\splash.png\\timetable.png\\timetable_fill.png\\tops_icon.png\\tshirt.png\\tweed.png\\user.png\\user_fill.png\\wardrobe.jpg\end{tabular} 
    & \begin{tabular}[c]{@{}l@{}}-\end{tabular} \\ \hline
    
    \begin{tabular}[c]{@{}l@{}}StyleReserve\\-Frontend\\/components/\end{tabular} & \begin{tabular}[c]{@{}l@{}}AddReservation.js\\Addclothfin.js\\Alarmcenterbanner.js\\DropdownList.js\\FindMyPW.js\\HeaderBar.js\\Logout.js\\Moreinfobanner.js\\NavigationBar.js\\NewReservation.js\\\end{tabular} 
    & \begin{tabular}[c]{@{}l@{}}Frontend (React Native)\end{tabular} \\
    \hline
    \begin{tabular}[c]{@{}l@{}}StyleReserve\\-Frontend\\/components/\end{tabular} & \begin{tabular}[c]{@{}l@{}}NumOfClothes.js\\PlusReservation.js\\ReserveDoneBanner.js\\ServiceLogo.js\\SignUpBanner.js\\SignUpPage.js\\StartWith.js\\addclothfin.js\\dropdownlist1.js\\signupbanner2.js\\\end{tabular} 
    & \begin{tabular}[c]{@{}l@{}}\\Frontend (React Native)\\\end{tabular} \\ \hline
    
    \begin{tabular}[c]{@{}l@{}}StyleReserve\\-Frontend\\/screens/\end{tabular} & \begin{tabular}[c]{@{}l@{}}AddClothesInfo.js\\EditInfo.js\\FindPassword.js\\FoundPassword.js\\Loading.js\\Login.js\\MainPage.js\\MyCloset.js\\MyPage.js\\OverallStylerReservation.js\\SignUpScreen.js\\StylerReservation.js\\TodayReservation.js\\layout.js\\personalstylingCalendar.js\\reserveclothes.js\end{tabular}
        & \begin{tabular}[c]{@{}l@{}}Frontend (React Native)\end{tabular} \\
    \hline
    \end{tabular}
\end{table}

4) Class Components:
\begin{itemize}
    \item Loading: It shows StyleReserve logo during the loading time before the application starts. Since the information of the user is unknown, after loading, the screen moves to the login page.\\
    
    \item Login: It helps users to log in to the service. It provides input fields for e-mail and password with the StyleReservation logo. When the user fills out the email and password fields and presses the ’Login’ button, the user input value is delivered to backend and frontend requests login, and when the backend successfully figured out that the user writes the correct information, then the user is delivered to Main page. At the bottom, there are 'Sign Up' and 'Find Password' buttons, so that users can get in to the service by setting some information, or find what the user set as password.\\
    
    \item Sign Up: If the user is first to use the service, it provides input fields of username, email, password, and password confirm. If the user wrote the email address that already exists in the account database or wrote different text in password and password confirm, then signing up is declined by the backend. If everything is correctly written down, then the backend stores the information, and the frontend shows the message that the user is successfully signed up, and move them to the login page.\\
    
    \item FindPassword: It helps users who forgot their password for some reasons to find their password that they wrote while signing up. It provides input fields of username and email, and if they wrote it correctly, then the backend check whether the matched information exists, and send the password to the email address that user wrote.\\
    
    \item FoundPassword: After the password is sent to the address, screen moves to this page, which says "Your password is sent to the email address that you wrote. Now, let's move to the login page". If the user clicks the button "Go back to Login Page", then the user comes back to the login page, and after checking password that the user forgot, user can successfully log in.\\
    
    \item MainPage: Main page shows the overall functions that the application have. There are two big blocks which shows today's, or recent schedule that user is going to have. The two buttons are for Styler and Styling Reservation. If the user press the button, then the user moves to the page that is clicked.\\
    
    \item Today Reservation: A quick look at the Styler bookings that have been booked throughout the day. Using the built-in time function, we receive today's date and basically show today's reservations. If you want to check reservations for other dates or add reservations, you can scroll through the calendar at the top to expand the size, and then select the desired date to move.\\
    
    \item MyCloset: Provides a service that helps users see their own clothes at a glance. Users can freely add the clothes they want to add through the button to enter new clothes, and press the existing clothes to set a date to wear the clothes and reserve the use of a styler.
    
    \item AddClothesInfo: Helps users enter information about clothes. If the user wants to add new clothes, go to the window and encourage the user to write information on the input window so that the user can enter the information on the clothes directly, or secure data through an API from Musinsa and store the clothes in a database.
    
    \item layout: basically shows the menu bar at the top and bottom required on the screen. The bar helps the user to move directly to the desired screen regardless of what screen the user is on. At the top, there are a back button and a notification button so that you can return to the previous screen or check the notification you received in real time. The bottom bar allows you to move to Home, Styler Reservation, Styling Reservation, and My Page, respectively.
    
    \item MyPage: You can check your name and email. In addition, if you want to stop using the service, you can log out of the service by pressing the 'Logout' button. If you press the logout button, you will be taken to the login page with a notification that you have successfully logged out.\\
\end{itemize}
    
5) Where It’s Taken From: The front end provides an interface that the user can see directly. The user enters the information required to use the StyleReserve service. In addition, based on the data stored in the database, it is possible to show the user the contents stored with various visual materials.\\

6) How / Why you used it
We used React Native with JavaScript. In addition, we continuously tested applications through the expo and created an intuitive UI/UX that is user-friendly. I wanted to create an application that could be used by both iPhone and Android phone users, and I had to be able to create services effectively in a high productivity and short time, and in the process, I chose the program and language.\\


\subsection{Module 2: Server}
1) Purpose: The backend server waits for requests from applications and manages and responds to the database, such as providing data read from the database when a function invokes the server. The values entered at the front end are stored in the database and managed continuously. The back-end server takes all the necessary data from the DB and delivers it to the React Native and StyleReserve applications.\\

2) Functionality: The server calculates, organizes, and stores information transmitted from the application in a database. In addition, when an application's function requests a specific value, it takes the necessary data from the database and returns it in addition to the request.\\

3) Location of Source Code: StyleReserve/StyleReserve-Backend\\
\begin{table} [h]
    \begin{tabular}{p{2cm}|p{3cm}|p{2.5cm}}
    \hline
    \textit{\textbf{Directory}} & \textit{\textbf{File Name}} & \textit{\textbf{Modules Used}}\\ \hline
    \begin{tabular}[c]{@{}l@{}}StyleReserve/\\StyleReserve\\-Backend\end{tabular} & \begin{tabular}[c]{@{}l@{}}index.js\\.sequelizerc\\package-lock.json\\package.json\\.envn\end{tabular} 
    & \begin{tabular}[c]{@{}l@{}}\\express, cookie-parser,\\dotenv, passport\\\end{tabular} \\
    
    \hline 
    \begin{tabular}[c]{@{}l@{}}StyleReserve\\-Backend\\/src/\end{tabular} & \begin{tabular}[c]{@{}l@{}}common/errorhandler.js\\common/resbuilder.js\\middleware/authenticate.js\\mockdata/users.js\\modules/passport.js\end{tabular} 
    & \begin{tabular}[c]{@{}l@{}}logger, jsonwebtoken,\\dotenv, passport\end{tabular} \\
    
    \hline 
    \begin{tabular}[c]{@{}l@{}}StyleReserve\\-Backend\\/src/routes/\\Auth\end{tabular} & \begin{tabular}[c]{@{}l@{}}authController.js\\authRouter.js\\authService.js\end{tabular} 
    & \begin{tabular}[c]{@{}l@{}}authController.js\\authRouter.js\\authService.js\end{tabular} \\
    
    \hline 
    \begin{tabular}[c]{@{}l@{}}StyleReserve\\-Backend\\/src/routes/\\Creserve\end{tabular} & \begin{tabular}[c]{@{}l@{}}crController.js\\crSearch.js\\crService.js\end{tabular} 
    & \begin{tabular}[c]{@{}l@{}}express, puppeteer,\\sequelize, logger,\\moment\end{tabular} \\
    
    \hline 
    \begin{tabular}[c]{@{}l@{}}StyleReserve\\-Backend\\/src/routes/\\Sreserve/\end{tabular} & \begin{tabular}[c]{@{}l@{}}srController.js\\srRouter.js\\srService.js\end{tabular} 
    & \begin{tabular}[c]{@{}l@{}}logger, express,\\sequelize, logger\end{tabular} \\
    
    \hline 
    \begin{tabular}[c]{@{}l@{}}StyleReserve\\-Backend\\/db/config\end{tabular} & \begin{tabular}[c]{@{}l@{}}config.js\end{tabular} 
    & \begin{tabular}[c]{@{}l@{}}path, dotenv\end{tabular} \\
    
    \hline 
    \begin{tabular}[c]{@{}l@{}}StyleReserve\\-Backend\\/db/migrations\end{tabular} & \begin{tabular}[c]{@{}l@{}}20221105111111-create\\-stylers.js\\20221105114511-create\\-users.js\\20221110075212-create\\-courses.js\\20221110075529-create\\-sreserves.js\\20221111052600\\-create-srmembers.js\\20221111063703-create\\-clothes.js\\20221111065544-create\\-creserves.js\end{tabular} 
    & \begin{tabular}[c]{@{}l@{}}-\end{tabular} \\
    
    \hline 
    \begin{tabular}[c]{@{}l@{}}StyleReserve\\-Backend\\/db/models\end{tabular} & \begin{tabular}[c]{@{}l@{}}clothes.js\\courses.js\\creserves.js\\sreserves.js\\srmembers.js\\stylers.js\\users.js\\index.js\end{tabular} 
    & \begin{tabular}[c]{@{}l@{}}sequelize\end{tabular} \\
    
    \hline 
    \begin{tabular}[c]{@{}l@{}}StyleReserve\\-Backend\\/config/\end{tabular} & \begin{tabular}[c]{@{}l@{}}logger.js\end{tabular} 
    & \begin{tabular}[c]{@{}l@{}}winston, winston-daily\\-roate-file, process,\\dotenv\end{tabular} \\
    
    \hline 
    \begin{tabular}[c]{@{}l@{}}StyleReserve\\-Backend\\/node_modules\end{tabular} & \begin{tabular}[c]{@{}l@{}}...\end{tabular} 
    & \begin{tabular}[c]{@{}l@{}}-\end{tabular} \\
    
    \hline
    \end{tabular}
\end{table}

4) Class component:
\begin{itemize}
    \item routes/Auth: It exchanges data related to user authentication such as membership registration and login. In the user authentication process, issue and use json web token. When logging in, the server issues a json web token to the client, and the client stores it and sends it to the server by including it in the body when requesting post for endpoint.\\
    
    \item routes/Sreserve: It is an endpoint responsible for importing Styler reservation information, adding and deleting new reservations, and adding and deleting existing reservations. The client sends the issued json web token to the body at login with other information, and the server authenticates the user through jwt and sends a response accordingly.\\
    
    \item routes/Creserve: It is an endpoint in charge of obtaining clothing information and clothing reservation information registered with Styler and adding clothing reservations. The module is responsible for not only clothing reservations, but also adding clothing to the database and additional clothing using the MUSINSA search engine. Likewise, user authentication is performed through the json web token.\\

\end{itemize}

5) Where it is taken from: The backend sends and receives data from the front end, and the database. The foundation is a server provided by AWS Lightsail, which is used in this project because it allows simple server operations at a low cost.\\

6) How and why we use it: By connecting users and databases through a back-end server, users could exchange information with databases through applications, and the server could connect all of them. As we decided to use JavaScript in React Native, we used node.js and express.js for flexible development.\\

\subsection{Module 3: Database}
1) Purpose: StyleReserve has a lot of data. There are Stylers, Users, Clothes, Styler Reservation, Clothes Reservation, etc., among which User must belong to only one Styler. In addition, this information should not be stored inside the user's mobile phone, but should be provided on a server occasionally with login information. We used the database to organize the data neatly, add the necessary information, and send it whenever requested.\\
Because the information we need is closely related to each other, we used SQL, a relational database, and we chose PostgreSQL, which is emerging these days, and we used a module called Sequelize to facilitate database operations in Nodejs.\\

2) Functionality: The server can retrieve information from the request and insert it into any table in the database via INSERT. Users can also receive the information they want through SELECT, delete the data through DELETE, or insert new data from existing data through UPDATE. You can also manage your data conveniently with a variety of features.\\

3) Location of Source Code : StyleReserve-Backend/ db/models/\\

4) Class Components:
\newpage
\begin{itemize}

    \item This is where you create the initial configuration files for the tables that will be generated by module sort.js. sequelize creates a database or tables in the language of javascript without creating a sql statement directly. The contents of this file must define the relationship between tables. You have to set what the primary key is and what the foreign key is, and you have to set whether it is an N:1 relationship or a 1:1 relationship. If you write it well including such contents, the sequelize module automatically creates and manages the database in postgreSQL. In the initial development test phase, we used sqlite as well as postgreSQL, which was very easy because it was possible to move by changing just one word.\\

    \begin{figure}[htbp]
    \centerline{\includegraphics[width=4cm]{stylers.png}}
    \label{fig}
    \caption{Styler Table}
    \end{figure}
    \item Stylers: The Styler table contains id, serial-num, auth-key. These three variables are essential to be registered in the Styler table. The id is primary Key and only one is given to one LG Styler Hardware. And this helps you identify Styler. Serial-num refers to a number that is uniquely assigned to each LG Stryer device and must be registered in the database.Auth-key helps you use it to identify the styler in the auth API and log in.\\
    
    \begin{figure}[htbp]
    \centerline{\includegraphics[width=4cm]{users.png}}
    \label{fig}
    \caption{User Table}
    \end{figure}
    \item User: The User table contains id, provider, password, name, nickname, salt, and styler-id. These six variables are essential to be registered in the Users table. The id is primary Key and helps you identify between different users. The password is used to log in to the account with an email, and the password is encrypted and stored, so it is impossible to change the value directly. The name is the name of the user and is used to identify the user information. nickname is the name that different users see.\\
    
    \begin{figure}[htbp]
    \centerline{\includegraphics[width=4cm]{clothes.png}}
    \label{fig}
    \caption{Clothes Table}
    \end{figure}
    \item Clothes: The Clothes table contains id, styler-id, name, brand-name, type. These five variables are essential to register Cloth in the Clothes table. The id is the primary key and is given only one per item. And it helps you identify clothes. The name literally means the product name or name of the clothes, and the user can write it down himself. Brand-name refers to the brand of the product that launched this clothing, and it must also be filled in by the user himself. Type refers to the type of clothing and plays a role in distinguishing whether it is a top or bottom.\\

    \begin{figure}[htbp]
    \centerline{\includegraphics[width=4cm]{creserve.png}}
    \label{fig}
    \caption{Creserve Table}
    \end{figure}
    \item Creserve: The Creserve table contains id, description, reservation-date, styler-id, user-id. These five variables are essential to register Cloth in the Clothes table. The id is the primary key and is given only one per item. And it helps you identify clothes. description means a brief description of this reservation and must be filled out by the user himself. The reservation-date is the part of the date you want to wear when you book your clothes. The same clothes cannot be booked on the same date. The styler-id and user-id are foreign keys. It helps you know which stylist's users made reservations.\\
    
    \begin{figure}[htbp]
    \centerline{\includegraphics[width=4cm]{sreserve.png}}
    \label{fig}
    \caption{Sreserve Table}
    \end{figure}
    \item Sreserve: The Sreserve table contains id, styler-id, course-id, owner-id, start-time. These five variables are essential to register Reservation with Styler. The start-time tells you when the Styler starts operating. This is an error if a different course-id is given at the same time in the same Styler because only one course runs in the same time zone per Styler.\\
    
    \begin{figure}[htbp]
    \centerline{\includegraphics[width=4cm]{courses.png}}
    \label{fig}
    \end{figure}
    \item Courses: Courses table includes id and duration. Here, duration means the time taken per each course, and the time taken according to each course is transmitted so that the user can use it in the process of making a reservation.\\
    
    \begin{figure}[htbp]
    \centerline{\includegraphics[width=4cm]{srmembers.png}}
    \label{fig}
    \caption{Srmembers Table}
    \end{figure}
    \item Srmembers: The Srmembers table includes id, sr-id, user-id, and count. Since the user can reserve a large number of clothes according to each styler, the number of clothes reserved through the count is stored.\\
\end{itemize}

5) How / Why you used it: Using the database, We were able to store all the users, clothes, and reservation information. The necessary information could be obtained through suitable queries. And, we were able to organize the contents of the table and draw specific information needed for the StyleReserve application.\\

\section{Use Cases}
\begin{figure}[htbp]
\centerline{\includegraphics[width=8cm]{UseCase.png}}
\label{fig}
\caption{Overall Usecase}
\end{figure}
The overall use case looks as follows. We found 9 specific use cases while using 'StyleReserve'.\\

\subsection{Use Case 1: Turn On the Application}
When the user clicks the 'StyleReserve' application icon, the application is activated, and the splash screen will be shown. If there are no histories of logging in, then after 3 seconds, the screen will move to the login screen. If the user was already logged in, then the screen directly goes to the Main Page, skipping the screens related to the account.\\
    
\subsection{Use Case 2: User Sign Up}
    \begin{figure}[htbp]
    \centerline{\includegraphics[width=6cm]{User Signup Process.png}}
    \label{fig}
    \caption{User Sign Up Process}
    \end{figure}
User creates an account by entering his or her information of [Username, Email, Password, and Password-Confirm]. If the email written is used before, signing up is not possible. Furthermore, if password and password-confim written are not same, signing up is not possible also. Only if the firstly-used email, password same with password-confirm is written, user can create an account. As creating an account succeed, then the data written goes to the database, and stores the new account information.\\
\begin{itemize}
    \item Username: The user enters the name to be used, which notifies a nickname within the application. If the name is correctly written without any issue, nothing pops up.
    \item Email: This e-mail information is used when logging into the application. If the user enters content that has some issue (ex. already used address, not in an email address type, etc.), sign up is not allowed.
    \item Password & Password-Confirm: This password information is used when logging in. The password has a rule of combining letterers less than 6 digits. There are also the space to write password one more time, in order to check the password is written in the right way. If password and password-confirm is not same, sign up cannot be possible and cannot move to the succeed page.\\
\end{itemize}
    
\subsection{Use Case 3: Find Password}
Users can find the password set when signing up if they forgot his or her password. When username and email is entered, they are send to the database whether related information is in, and if the existence of the user is valid, then the password set before is send to the email address that the user wrote.\\

\subsection{Use Case 4: Log In}
    \begin{figure}[htbp]
    \centerline{\includegraphics[width=5cm]{User Login Process.png}}
    \label{fig}
    \caption{User Login Process}
    \end{figure}
The user must enter their e-mail(ID) and password in the email and password fields for Login, respectively. The user enters the information in the fields and then press the login button. If the user’s email and password are established, it is approved and allow the user to move to the main page. If the email and password are not established, the message “login failed” appears, and the user should try another method to log in.\\
    
\subsection{Use Case 5: Main page - Daily Check}
    \begin{figure}[htbp]
    \centerline{\includegraphics[width=4cm]{MainPage Process.png}}
    \label{fig}
    \caption{Main Page Process}
    \end{figure}
This is the main page of StyleReserve. The user can check the overall information on the application.\\

\newpage
\begin{itemize}
    \begin{figure}[htbp]
    \centerline{\includegraphics[width=5cm]{main_page_process.png}}
    \label{fig}
    \caption{Main Page Top Box}
    \end{figure}
    \item Today’s status: This is a box spaces on the top that allows users to check recently planned reservations, composed with Styler and Styling reservation. When new reservation is made and it is the most upcoming event, the information is changed into new reservation. Boxes can’t be clicked, and they are only for the viewing purpose.\\

     \begin{figure}[htbp]
    \centerline{\includegraphics[width=5cm]{styler_reservation_button.png}}
    \label{fig}
    \caption{Styler Reservation Button}
    \end{figure}
    \item Styler Reservation: In Styler Reservation item, users is provided a big button which helps the user to experience Styler reservation. When the user clicks on this item, it goes to specific Styler reservation page.\\
    
    \begin{figure}[htbp]
    \centerline{\includegraphics[width=5cm]{styling_reservation_button.png}}
    \label{fig}
    \caption{Styling Reservation Button}
    \end{figure}
    \item Styling Reservation: In Styling Reservation item, users is provided a big button which helps the user to experience Styling reservation. When the user clicks on this item, it goes to specific styling reservation page.\\
\end{itemize}

\subsection{Use Case 6: Reserve Styler}
At Styler Reservation, users make reservations based on the time they want to use the styler, the course they want to use, and the number of clothes to use the styler. You can basically check the reservations you received by passing today's date to the Database, and if you want to see a reservation for another date, you can select a date from the calendar. Reservations can be made in two ways depending on the reservation status.\\
\begin{itemize}
    \begin{figure}[htbp]
    \centerline{\includegraphics[width=5cm]{add_reserve.png}}
    \label{fig}
    \caption{Adding Reservation Process}
    \end{figure}
    \item The first reservation method is to add the user's reservation to the existing reservation. In this case, it is possible when there is a seat on the hanger left with the reservation made for the time and course I wanted. When the user enters the number of clothes to reserve, the data is transferred to the database, and on the screen, the number increases by the number of clothes added by the user and the user's name appears in the reservation.\\
    
    \begin{figure}[htbp]
    \centerline{\includegraphics[width=5cm]{new_reserve.png}}
    \label{fig}
    \caption{Making New Reservation Process}
    \end{figure}
    \item The second way to make a reservation is to make a new reservation. This process requires thinking about setting a new time, what course to use, and how many clothes to use in the stylist. The second method is done when the user does not have the desired time or course in the existing reservation, or when the user is already fully booked and cannot be added. Also, when you press Reservation, the contents automatically appear on the screen, so that all people who share the same Styler can check the reservation and proceed with additional reservations.\\
\end{itemize}

\subsection{Use Case 7: Add new cloth to My Closet}
My Closet brings up the closet of the styler stored in the database and shows it to the user through the front end. If you have any clothes you want to add in your wardrobe, you will send them to the backend by filling in the input fields such as the clothing's name, brand, item, etc. Once the product has been successfully added to the database, you can proceed if the product appears on the screen and you wish to schedule a clothing styling schedule in the future.\\

\subsection{Use Case 8: Reserve Styling}
Users can select the clothes they want to reserve for styling in the closet and then move on to the styling reservation system. Styling Schedules, like Styling Schedules, call up a calendar to prompt users to view the schedule history. If a reservation already exists on the date selected by the user, it is designed not to be able to automatically make a reservation. The user writes down the reason for booking the clothes, and the contents are sent to the database and the reservation is successful.\\

\begin{itemize}

    \begin{figure}[htbp]
    \centerline{\includegraphics[width=5cm]{sr_2.png}}
    \label{fig}
    \caption{Overall Styling Calendar}
    \end{figure}
    \item This is a calendar that shows the total date of the reservation for all the clothes in My Closet. On the day of the reservation, dots are marked below the number, which allows the user to check the approximate status of the reservation. The button at the bottom of the calendar goes to My Closet and helps me check the clothes I have.\\
    
    \begin{figure}[htbp]
    \centerline{\includegraphics[width=4.6cm]{select_cloth.png}}
    \label{fig}
    \caption{Example Button for Each Cloth}
    \end{figure}
    \item Users can scroll through My Closet's clothes and choose the clothes they want to reserve. Each picture of clothing is a type of button, and if the user clicks on a particular clothing, the user can go to the reservation window for the clothing.\\
    
    \begin{figure}[htbp]
    \centerline{\includegraphics[width=5cm]{sr_3.png}}
    \label{fig}
    \caption{Styling Reserving Process}
    \end{figure}
    \item When the user selects a specific outfit, a calendar appears to help the user select the date he or she wants to reserve. The calendar can be moved to the previous and subsequent months through the arrows on both sides. The date on which the reservation has already been made will not be selected as the gray circle is laid against the background. The date the user clicked on is covered with a red circle in the background and helps the user see that they have chosen.\\
    
    \begin{figure}[htbp]
    \centerline{\includegraphics[width=5cm]{sr_4.png}}
    \label{fig}
    \caption{Select Date Box}
    \end{figure}
    \item When a date is clicked, the text at the bottom of the calendar changes from 'Please select a date' to the actual date (e.g. 08/12/2022), which allows the user to check the date he/she chooses through text in addition to changing the numeric background color on the calendar.
    The text box at the bottom is a space that allows users to write the reason for booking clothes, and is designed to prepare for situations where users forget the reason for booking clothes.\\
    
    \newpage
    \begin{figure}[htbp]
    \centerline{\includegraphics[width=5cm]{sr_5.png}}
    \label{fig}
    \caption{Reservation Success Notification}
    \end{figure}
    \item After the user writes the reason for the reservation through the keyboard input, the 'Register Schedule' button is pressed. When the button is pressed, a message 'Schedule Registration Successful' appears, indicating that the reservation has been successfully made. Through this, the user can verify that the process just now went smoothly.\\
    
    \begin{figure}[htbp]
    \centerline{\includegraphics[width=5cm]{sr_6.png}}
    \label{fig}
    \caption{Reservation Notification Box}
    \end{figure}
    \item In the calendar of the "Overall Styling Reservation" screen, which we saw at the beginning, a dot appeared at the bottom of the number of the date the user newly reserved, indicating that the reservation was successful. By pressing the corresponding button, the user can check the name of the clothing, the photo of the clothing, and the purpose of the clothing reservation. These reservation details can be checked not only by the party who made the reservation, but also by those who share the same wardrobe, i.e., those who use the same styler, preventing duplicate reservations of the same clothes.\\
    
\end{itemize}

\newpage
\subsection{Use Case 9: Notification}
\begin{itemize}
    \begin{figure}[htbp]
    \centerline{\includegraphics[width=5cm]{notification_1.png}}
    \label{fig}
    \caption{Notification Button}
    \end{figure}
    \item All alerts that occur within the app can be checked by clicking the notification icon at the top right of each page. The user can check anytime for what notification the user got.\\
    
    \begin{figure}[htbp]
    \centerline{\includegraphics[width=5cm]{notification_2.png}}
    \label{fig}
    \caption{Notification Box}
    \end{figure}
    \item Users can receive notifications of their own scheduled styling through the banner. On the day before the reservation, the database delivers the data the day before the reservation to the front desk, and the front desk that receives the data sends the notification to the user.\\
\end{itemize}

\section{Software Installation Guide}
User can download ’StyleReserve’ on Google Playstore of Android or App Store of iOS. Users can search our application with keywords such as 'Styler', 'LG', 'Styling', 'Reservation', and
etc. When the user presses the ’download’ button, ’StyleReserve’
will be installed in user’s mobile phone.\\

\section{Conclusion}
The 'StyleReserve' application provides a Styler reservation service that allows a large number of users to use the Styler at any time they want, and to wear neat clothes to important appointments, taking into consideration the purpose and environment of using LG Styler.\\
In the course of competitive comparisons with third-party products, we found that our products offer consumers overwhelming capabilities through differentiated patent technologies, but leave consumers somewhat to be desired in terms of software. Based on this, we have come up with the following ideas to improve software functionality to provide consumers with a satisfactory process from technology to service.\\
During the online research process, I found that there were many difficult situations due to overlapping clothes that I wanted to wear when I shared the same closet with my acquaintances. In addition, we predicted that Styler would not just be used at home, but when it was used in public spaces such as offices, it would be a problem with a large number of users.\\
Accordingly, we planned two functions - Styler reservation and styling reservation. The Styler Reservation feature was designed to focus more on Styler environments, reflecting the course, reservation time, and the number of clothes to help save time and power by encouraging users to use similar courses together. The Style Booking feature focused on families sharing the same wardrobe, alerting them to use the Styler the day before when they book clothes on a specific day, prompting them to wear neat clothes and preventing them from wearing the same clothes on the same day.\\
Styler is now becoming a daily necessity, not a luxury. It is the most popular product as a honeymoon gift and the home appliance that the MZ generation is currently paying attention to. With LG Styler, which is loved for its outstanding technology, we are confident that more people will be able to use it in a more diverse environment and have a satisfying experience if features that reflect user insights are added.\\

\section{Discussion}
In the initial planning process, we tried to include not only "reservation" but also "recommendation" function based on Instagram hashtags. However, considering the level of teams that are mostly new to development, it was practically difficult to challenge that function. As a result, it is regrettable that we could not create an application that could have various functions.\\
However, I learned about developing applications through this opportunity, and as I have more time and opportunities in the future, I wanted to study more about it and create an application that reflects all the features I wanted.\\
In addition, the existing plan was to consider voice recognition services for users who were busy preparing for work and school. It was judged that the Styler is generally placed in the closet, so it has a great impact on the process of wearing clothes, and accordingly, after installing a display outside the Styler, it was intended to induce data such as weather, traffic information, and time through voice recognition linkage. However, the fact that related ideas were actually carried out within LG Electronics, and that there was a little limit to considering displays with our team's technology, and unfortunately, we had no choice but to give up the idea.\\
If I have a chance, I would like to develop software that can help the overall user's direct and indirect Styler use environment by reflecting the function in the Styler.\\

\end{document}
